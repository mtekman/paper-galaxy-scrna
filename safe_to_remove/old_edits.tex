% file of old edits

%\newcommand{\citeneed}{{\bf\tiny [Citation Needed]}}

%\item Galaxy provides rich scRNA-seq workflows and tutorials for the processing of both 10x and non-10x datasets.

% 
%The various expression profiles uncovered within tissue samples infer discrete cell types which are related to %one another across an ``expression landscape'', where relationships between the more distinct profiles are %inferred via distance-metrics or manifold learning techniques which aim to model the continuous biological %process of cell differentiation from multipotent stem cells to mature cell types, and infer lineage and %differentiation pathways between these various distinct and transient cell types \cite{wagner2016revealing}.

%\paragraph{Manifold of protocols and their problems (unclear) + Normalization}

transcript, or that the sequencer %simply did not detect them. Traditional normalisation techniques such as \prog{EdgeR} and \prog{Limma} had to 
simply did not detect them. Normalisation techniques originally developed for

% cite ScanPy, Seurat here?

%\paragraph{pipelines versus single approaches}

%\paragraph{10x Rise}

From the user perspective, literate datasets no longer needed to be analysed in RStudio but were now permitted in less language-restrictive notebooks such as Jupyter\citep{allaire2012rstudio,kluyver2016jupyter}.

%\paragraph{terminology - Pipeline, Suite, Workflow, which to use?}

%\paragraph{ScanPy Integration and Dominance}
%One of the first packages to make use of this language independence was the ScanPy suite, written in Python, a more widely %used language \citep{stackoverflow2019,wolf2018scanpy}. Within a year, ScanPy had become the dominant suite for analysing 10x %datasets, and quickly adopted the \fileformat{HDF5} format, with their own \fileformat{AnnData} extension better suited to %their pipeline. The Seurat developers had similar aspirations and soon adopted the \fileformat{LOOM} format, another %\fileformat{HDF5} variant. However, the dominance of ScanPy became much more pronounced as it began to rapidly integrate the %methods of other standalone packages into its codebase, making it the natural choice for users who wanted to achieve more %without compromising on compatibility between different suites \citep{scanpyreleases}.

%The Galaxy project provides a priority processing queue for anybody who may require it during a workshop.

%In each protocol, cells are tagged with cell barcodes such that any reads derived from them can be unambiguously quantified to originate from a specific cell, and the inclusion of unique molecular identifiers (UMIs) are also employed to mitigate the effects of amplification bias of transcripts within the same cell. The detection, extraction, and (de-)multiplexing of these barcodes is therefore one of the first hurdles researchers encounter when receiving their raw FASTQ data from the sequencing facility.

%continue to outlive other training resources which are known to decay once interest in them declines. 


\paragraph{Improvements in sequencing}
The sequencing technologies also aspired to solve the capture efficiency via well, droplet, and flow cytometry based protocols, all of which lend importance to the process of barcoding sequencing reads. 

In each protocol, cells are tagged with cell barcodes such that any reads derived from them can be unambiguously assigned to the cell of origin. The inclusion of unique molecular identifiers (UMIs) are also employed to mitigate the effects of amplification bias of transcripts within the same cell. The detection, extraction, and (de-)multiplexing of cell barcodes and UMIs is therefore one of the first hurdles researchers encounter when receiving raw FASTQ data from a sequencing facility.


\paragraph{Sequencing sensitivity and Normalization}
With each new protocol comes a host of new technical problems to overcome. The first wave of software utilities to deal with the analysis of single cell datasets were statistical packages, aimed at tackling the issue of the dropout events that occured during sequencing, which would manifest as a high prevalence of zero-entries in over 80\% of the feature-count matrix. These zeroes were problematic, since they could not be trivially ignored as their presence stated that either the cell did not produce any molecules for that transcript, or that the sequencer simply did not detect them. Normalisation techniques originally developed for bulk RNA-seq had to be adapted to accommodate for this uncertainty, and new ones were created that harnessed hurdle models, data imputation via manifold learning techniques, or by pooling subsets of cells together and building general linear models \citep{camara2018methods}.

